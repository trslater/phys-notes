\documentclass{article}

\usepackage{amsmath}
\usepackage[hidelinks]{hyperref}
\usepackage{booktabs}

\title{Translational to Angular Conversions}

\begin{document}
    \maketitle

    \section{Lookup}
    
    \begin{table}[h]
        \centering
        \begin{tabular}{c|cl}
            $x$ & $\theta$ \\
            $v$ & $\omega$ \\
            $a$ & $\alpha$ \\
            $m$ & $I$ & Moment of inertia \\
            $F$ & $\tau$ & Torque \\
            $p$ & $L$ & Angular momentum \\
        \end{tabular}
    \end{table}
    
    \section{Basic Conversions}

    \begin{equation}
        \label{eq:moment-of-inertia}
        I = \sum{mr^2}
    \end{equation}

    \begin{align}
        s &= r\theta \\
        \label{eq:angular-velocity}
        v &= r\omega = r\dot{\theta} \\
        a &= r\alpha = r\dot{v} = r\ddot{\theta}
    \end{align}
    
    \begin{align}
        \label{eq:angular-momentum}
        L &= I\omega = rp \\
        \tau &= I\alpha = rF
    \end{align}

    \subsection{Understanding Angular Momentum}
    Starting with \autoref{eq:angular-momentum} and using \autoref{eq:moment-of-inertia} and \autoref{eq:angular-velocity}: $$\begin{aligned}
        L &= I\omega \\
        &= \frac{v}{r}\sum{mr^2} \\
        &= vmr, & \text{assuming a single point object} \\
        &= rp
    \end{aligned}$$

    \section{Vectors}

    \begin{equation}
        \vec{L} = \vec{r} \times \vec{p}
    \end{equation}
\end{document}
